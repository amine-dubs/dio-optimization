\documentclass[11pt, a4paper]{article}
\usepackage[utf8]{inputenc}
\usepackage{geometry}
\usepackage{hyperref}
\usepackage{enumitem}

\geometry{
    a4paper,
    left=2.5cm,
    right=2.5cm,
    top=2.5cm,
    bottom=2.5cm
}

\title{Author Information and Affiliations}
\author{}
\date{}

\begin{document}

\maketitle

\section*{Authors}

\subsection*{1. Mohamed Amine Bellatreche (Corresponding Author)}

\textbf{Affiliation:} \\
Department of Computer Science \\
University of Science and Technology of Oran Mohamed Boudiaf (USTO-MB) \\
BP 1505, El M'Naouer, Oran 31000, Algeria

\textbf{Contact Information:} \\
Email: amine.bellatreche@univ-usto.dz \\
Alternative Email: m.bellatreche@etu.univ-usto.dz \\
ORCID: [To be added] \\
Phone: [To be added]

\textbf{Biographical Sketch:} \\
Mohamed Amine Bellatreche is a Computer Science researcher at the University of Science and Technology of Oran Mohamed Boudiaf (USTO-MB), Algeria. His research focuses on metaheuristic optimization algorithms, machine learning, and their applications to medical diagnostics and computer vision. He specializes in nature-inspired optimization, feature selection, and cross-domain validation of intelligent systems. His work emphasizes rigorous statistical validation and deployment-ready solutions for real-world problems.

\textbf{Research Interests:}
\begin{itemize}[leftmargin=*]
    \item Metaheuristic optimization algorithms (nature-inspired, swarm intelligence)
    \item Feature selection and dimensionality reduction
    \item Hyperparameter optimization and AutoML
    \item Medical machine learning and diagnostic systems
    \item Computer vision and deep learning optimization
    \item Cross-domain validation and transfer learning
    \item Statistical validation and reproducible research
\end{itemize}

\textbf{Contributions to This Work:}
\begin{itemize}[leftmargin=*]
    \item Conceptualization and research design
    \item Complete Python implementation of DIO algorithm from MATLAB specification
    \item Development of nested optimization framework
    \item Implementation of 30-run statistical validation protocol
    \item Formal analysis: Wilcoxon signed-rank tests, confidence intervals
    \item Breast cancer classification experiments (single-split, CV-based, XGBoost)
    \item CIFAR-10 computer vision extension experiments
    \item Discovery and characterization of algorithm-dependent optimization overfitting
    \item Writing—original draft preparation
    \item Creation of all visualizations (24 figures, 8 tables)
    \item Code documentation and repository management
\end{itemize}

\textbf{Education:}
\begin{itemize}[leftmargin=*]
    \item Master's Student in Data Science, USTO-MB, Algeria (Expected: [Year])
    \item Bachelor's Degree in Computer Science, USTO-MB, Algeria ([Year])
\end{itemize}

\textbf{Publications:}
\begin{itemize}[leftmargin=*]
    \item [List previous publications if any, otherwise state "First major publication"]
\end{itemize}

\vspace{1cm}

\subsection*{2. Ghizlane Cherif (Co-Author)}

\textbf{Affiliation:} \\
Data Science Speciality \\
University of Science and Technology of Oran Mohamed Boudiaf (USTO-MB) \\
BP 1505, El M'Naouer, Oran 31000, Algeria

\textbf{Contact Information:} \\
Email: ghizlane.cherif@univ-usto.dz \\
Alternative Email: [To be added] \\
ORCID: [To be added]

\textbf{Biographical Sketch:} \\
Ghizlane Cherif is a Data Science researcher at the University of Science and Technology of Oran Mohamed Boudiaf (USTO-MB), Algeria. Her expertise spans machine learning methodology, optimization algorithm validation, and practical deployment of intelligent systems. She specializes in fitness function design, cross-validation strategies, and resource management for computationally intensive optimization tasks. Her work emphasizes bridging the gap between theoretical algorithm development and real-world clinical applications.

\textbf{Research Interests:}
\begin{itemize}[leftmargin=*]
    \item Machine learning methodology and best practices
    \item Cross-validation and model validation strategies
    \item Fitness function design for multi-objective optimization
    \item Medical AI and clinical decision support systems
    \item Transfer learning and deep feature optimization
    \item Computational resource optimization
    \item Reproducible research and statistical rigor
\end{itemize}

\textbf{Contributions to This Work:}
\begin{itemize}[leftmargin=*]
    \item Methodology: Fitness function design (Equation 1: accuracy + feature penalty)
    \item Methodology: CV-based optimization strategy design (Equation 8: k-fold fitness)
    \item Validation: Cross-domain experimental validation framework
    \item Resources: Dataset curation and preprocessing
    \item Resources: ResNet50 feature extraction using Google Colab GPU
    \item Investigation: CIFAR-10 model selection experiments (XGBoost vs. RF vs. others)
    \item Analysis: Comparative analysis of medical vs. vision domain results
    \item Writing—Review \& Editing: Critical revision for intellectual content
    \item Supervision: Research methodology oversight and quality assurance
\end{itemize}

\textbf{Education:}
\begin{itemize}[leftmargin=*]
    \item [Degree] in Data Science, USTO-MB, Algeria ([Year])
    \item [Previous degrees if applicable]
\end{itemize}

\textbf{Publications:}
\begin{itemize}[leftmargin=*]
    \item [List previous publications if any, otherwise state "First major publication"]
\end{itemize}

\vspace{2cm}

\section*{Institutional Affiliation Details}

\subsection*{University of Science and Technology of Oran Mohamed Boudiaf (USTO-MB)}

\textbf{Official Name:} Université des Sciences et de la Technologie d'Oran Mohamed Boudiaf

\textbf{Location:} \\
BP 1505, El M'Naouer \\
Oran 31000 \\
Algeria

\textbf{Department:} Computer Science

\textbf{Website:} \url{https://www.univ-usto.dz}

\textbf{About the Institution:} \\
USTO-MB is a leading Algerian university specializing in science, technology, and engineering. The Department of Computer Science conducts research in artificial intelligence, machine learning, data science, and computational optimization. The university emphasizes practical applications of theoretical research to address real-world challenges in healthcare, industry, and society.

\vspace{2cm}

\section*{Funding and Acknowledgments}

\textbf{Funding Sources:} \\
This research received no specific grant from any funding agency in the public, commercial, or not-for-profit sectors. All computational resources were provided by the authors' institutional affiliations or free cloud computing platforms (Google Colab).

\textbf{Acknowledgments:}
\begin{itemize}[leftmargin=*]
    \item We acknowledge the developers of the original DIO algorithm, Dehghani et al., for their innovative work on nature-inspired optimization published in \textit{Scientific Reports} (2023).
    \item We thank the UCI Machine Learning Repository for maintaining the Breast Cancer Wisconsin (Diagnostic) dataset.
    \item We acknowledge the open-source communities behind Python, Scikit-learn, XGBoost, TensorFlow/Keras, NumPy, Pandas, Matplotlib, and related libraries that made this research possible.
    \item We thank Google Colab for providing free GPU resources for ResNet50 feature extraction.
\end{itemize}

\vspace{2cm}

\section*{Ethics and Competing Interests}

\textbf{Ethics Statement:} \\
This research used publicly available, de-identified datasets that do not require ethics approval:
\begin{itemize}[leftmargin=*]
    \item \textbf{Breast Cancer Wisconsin (Diagnostic):} Public dataset from UCI Machine Learning Repository, originally collected with patient consent at University of Wisconsin Hospitals (Street et al., 1993).
    \item \textbf{CIFAR-10:} Public image dataset with no human subjects (Krizhevsky \& Hinton, 2009).
\end{itemize}

No new human subjects data were collected for this study.

\textbf{Competing Interests:} \\
The authors declare no competing interests, financial or otherwise, related to this research.

\vspace{2cm}

\section*{Data and Code Availability}

\textbf{Datasets:}
\begin{itemize}[leftmargin=*]
    \item \textbf{Breast Cancer Wisconsin (Diagnostic):} Available at UCI Machine Learning Repository \\
    \url{https://archive.ics.uci.edu/ml/datasets/Breast+Cancer+Wisconsin+(Diagnostic)}
    \item \textbf{CIFAR-10:} Available at University of Toronto \\
    \url{https://www.cs.toronto.edu/~kriz/cifar.html}
\end{itemize}

\textbf{Code Repository:} \\
Python implementation of DIO algorithm, nested optimization framework, and all experimental code will be made publicly available on GitHub upon manuscript acceptance: \\
\url{https://github.com/amine-dubs/dio-optimization}

Repository will include:
\begin{itemize}[leftmargin=*]
    \item Complete DIO Python implementation
    \item Nested optimization framework code
    \item Breast cancer classification experiments (30-run protocol)
    \item CIFAR-10 computer vision experiments
    \item Statistical analysis scripts (Wilcoxon tests, visualizations)
    \item Requirements.txt for reproducibility
    \item Comprehensive README with usage instructions
    \item Jupyter notebooks demonstrating key experiments
\end{itemize}

\textbf{License:} MIT License (to be specified upon repository creation)

\vspace{2cm}

\section*{Corresponding Author Responsibilities}

Mohamed Amine Bellatreche, as the corresponding author, will:
\begin{itemize}[leftmargin=*]
    \item Handle all communication with the journal editor and reviewers
    \item Coordinate responses to peer review comments with co-author
    \item Ensure accuracy of all submitted materials
    \item Manage revisions and resubmissions
    \item Coordinate publication logistics (proofs, copyright, open access fees if applicable)
    \item Serve as primary point of contact for post-publication inquiries
    \item Maintain the public code repository
\end{itemize}

\vspace{1cm}

\noindent
\textbf{Contact for Correspondence:} \\
Mohamed Amine Bellatreche \\
Email: amine.bellatreche@univ-usto.dz \\
Department of Computer Science \\
University of Science and Technology of Oran Mohamed Boudiaf (USTO-MB) \\
Oran, Algeria

\end{document}
